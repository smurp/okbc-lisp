\chapter{The GFP Knowledge Model}
\label{ch:reference-model}

The GFP protocol provides operations for knowledge expressed in an
implicit representation formalism called the "GFP Knowledge Model",
which we will specify in this chapter.  The GFP Knowledge Model supports
an object-oriented representation of knowledge and provides a set of
representational constructs commonly found in frame representation
systems (FRS's) \cite{KarpFreview}.  The GFP Knowledge Model serves as
an implicit interlingua for knowledge communication in that knowledge
obtained via GFP from a knowledge representation system (KRS) or
provided to a KRS via GFP is expressed in the Knowledge Model's
representational constructs.

The GFP Knowledge Model consists of knowledge bases, frames, slots,
facets, classes, and constants.  We describe each of these constructs in
the sections below.


\section{Entities}

The GFP Knowledge Model assumes a universe of discourse consisting of
all entities about which knowledge is to be expressed.  Each use of GFP
is likely to have a different universe of discourse.  However, GFP
assumes the universe of discourse always includes all constants of the
following {\em basic types}:

\bitem

\item integers
\item numbers
\item strings
\item symbols
\item lists of entities
\item sets of entities

\eitem

Symbols are used as names of entities.  Sets are also referred to as
{\em classes}\footnote{We use the term {\em class} synonymously with the
term {\em concept} as used in the description logic community.}.  We
refer to entities that are not classes as {\em individuals}.  Thus, the
domain of discourse is considered to consist of individuals and classes.


\section{Frames, Own Slots, and Own Facets}

A {\em frame} is a structured object that represents an entity in the
domain of discourse.  Formally, a frame corresponds to a non-logical
symbol.  A frame that represents a class is called a {\em class frame},
and a frame that represents an individual is called an {\em individual
frame}.

A frame has associated with it a collection of {\em own slots}, and each
own slot has associated with it a set of entities called {\em slot
values}.  Formally, a slot is a binary relation, and each value V of an
own slot S in a frame F represents the assertion that the relation S
holds for the entity represented by F and the entity represented by V
(i.e., "S(F V)").  For example, the assertion that Fred's favorite foods
are potato chips and ice cream could be represented by the {\tt
favorite-food} own slot in the {\tt fred} frame having as values the
{\tt potato-chips} frame and the string {\tt "ice cream"}.

Relations may optionally be considered to be entities in the domain of
discourse and therefore representable by frames.  Thus, a slot can
optionally be represented by a frame.  Such a frame would describe the
properties of the relation represented by the slot.  Slots and frames
necessarily share the same name space since a given name can refer to
both a slot and to the frame representing that slot as a relation.

An own slot on a frame has associated with it a collection of {\em own
facets}, and each own facet has associated with it a set of entities
called {\em facet values}.  Formally, a facet is a ternary relation, and
each value V of own facet Fa on slot S in frame Fr represents the
assertion that the relation Fa holds for the entity represented by Fr,
the relation S, and the entity represented by V (i.e., "Fa(Fr,S,V)"). 
For example, the assertion that the favorite foods of Fred must be
edible foods could be represented by the facet {\tt value-type} on the
{\tt favorite-food} slot in the {\tt fred} frame having the value {\tt
edible-food}.


\section{Classes and Instances}
\label{intro:classes}

A {\em class} is a set of entities. Each of the entities in a class is
said to be an {\em instance} of the class.  An entity can be an instance
of many classes, which are called its {\em types}.  A class can be an
instance of a class of classes.  A class which has instances that are
themselves classes is called a {\em meta-class}.

The class membership relation (called {\em instance-of}) that holds
between an instance and a class is a binary relation that maps entities
to sets.  The relation {\em type-of} is defined as the inverse of {\em
instance-of}, as follows:

\begin{verbatim}
     (<=> (type-of $C $I) (instance-of $I $C))
\end{verbatim}

The {\em subclass-of} relation for classes is defined in terms of {\em
instance-of}, as follows.  A class $C_{sub}$ is a subclass of class
$C_{super}$ if and only if all instances of $C_{sub}$ are also instances
of $C_{super}$.  That is:

\begin{verbatim}
     (<=> (subclass-of $Csub $Csuper)
          (forall ($I) (=> (instance-of $I $Csub)
                           (instance-of $I $Csuper))))
\end{verbatim}

Note that this definition implies that {\em subclass-of} is transitive. 
(I.e., If A is a subclass of B and B is a subclass of C, then A is a
subclass of C.)

The relation {\em superclass-of} is defined as the inverse of {\em
subclass-of}, as follows:

\begin{verbatim}
     (<=> (superclass-of $Csuper $Csub) (subclass-of $Csub $Csuper))
\end{verbatim}


\section{Class Frames, Template Slots, and Template Facets}

A class frame has associated with it a collection of {\em template
slots} that describe own slot values considered to hold for each member
of the class represented by the frame.  The values of template slots are
said to {\em inherit} to the subclasses and to the instances of a
class.  Formally, each value V of a template slot S in a class frame C
represents the assertion that the relation {\em template-slot-value-of}
holds for the class represented by C, the relation S, and the entity
represented by V (i.e., "template-slot-value-of(C,S,V)").  That
assertion, in turn, implies the relation S holds between each instance I
of class C and value V (i.e., "S(I,V)").  It also implies that the
relation {\em template-slot-value-of} holds for each subclass Csub of
class C, the relation S, and the entity represented by V (i.e.,
"template-slot-value-of(Csub,S,V)").  In general, the following axioms
hold for the relation {\em template-slot-value-of}: 

\begin{verbatim}
     (=> (template-slot-value-of $C $S $V)
         (=> (instance-of $I $C) (holds $S $I $V)))

     (=> (and (template-slot-value-of $Csuper $S $V)
              (subclass-of $Csub $Csuper)) 
         (template-slot-value-of $Csub $S $V))
\end{verbatim}

Thus, the values of a template slot are inherited to subclasses as
values of the corresponding template slot and to instances as values of
the corresponding own slot.

A template slot on a class frame has associated with it a collection of
{\em template facets} that describe own facet values considered to hold
for the corresponding own slot on each member of the class represented
by the class frame.  As with the values of template slots, the values of
template facets are said to inherit to the subclasses and to the
instances of a class.  Formally, each value V of a template facet F on a
template slot S in a class frame C represents the assertion that the
relation {\em template-facet-value-of} holds for the class represented
by C, the relations S and F, and the entity represented by V (i.e.,
"template-facet-value-of(C,S,F,V)").  That assertion, in turn, implies
the relation F holds for each instance I of class C, relation S, and
value V (i.e., "F(I,S,V)"). It also implies that the relation {\em
template-facet-value-of} holds for each subclass Csub of class C, the
relations S and F, and the entity represented by V (i.e.,
"template-facet-value-of(Csub,S,F,V)").  For example, the assertion that
the gender of all female persons is female could be represented by
template slot {\tt gender} on class frame {\tt female-person} having the
value {\tt female}.  Then, if we created an instance of {\tt
female-person} called {\tt mary}, {\tt female} would be a value of the
own slot {\tt gender} on {\tt mary}.

In general, the following axioms hold for the relation {\em
template-facet-value-of}: 

\begin{verbatim}
     (=> (template-facet-value-of $C $S $F $V)
         (=> (instance-of $I $C) (holds $F $I $S $V)))

     (=> (template-facet-value-of $Csuper $S $F $V)
         (=> (subclass-of $Csub $Csuper)
             (template-facet-value-of $Csub $S $F $V))
\end{verbatim}

Thus, the values of a template facet are inherited to subclasses as
values of the corresponding template facet and to instances as values of
the corresponding own facet.


%It is often important to be able to retrieve the {\em direct} instances,
%subclasses, and superclasses of a given class from a knowledge base. 
%The {\em direct instances} of a class $C$ are those entities that are
%instances of $C$ and not known to be instances of any subclass of $C$ in
%the knowledge base.  The {\em direct subclasses} of a class $C$ are
%those entities that are subclasses of $C$ and not known to be subclasses
%of any subclass of $C$ in the knowledge base. The {\em direct
%superclasses} of a class $C$ are those entities that are superclasses of
%$C$ and not known to be superclasses of any superclass of $C$ in the
%knowledge base.

%when retrieving instance-of and subclass-of relationships from a
%knowledge base to distinguish between those relationships that are
%explicitly stored in the knowledge base (in whatever form the knowledge
%representation system stores such relationships) and those that are
%derivable using the definition of {\em subclass-of}.  The GFP protocol
%provides operations for retrieving such explicit relationships and
%refers to them as {\em direct instances}, {\em direct subclasses}, and
%{\em direct superclasses}.  Thus,   The GFP protocol provides for that
%distinction by referring to the instances of a class that are explicitly
%stored with a class are referred to in the GFP protocol as {\em direct
%instances of} the class, and the subclasses of a class that are
%explicitly stored with a class are referred to as {\em direct
%subclasses. asserted to be a  individual frame is a direct-instance-of a
%particular class $C$, and where the instance might be a
%direct-instance-of some other class $C$, that in turn is a sub of $C$. 
%We use the relations {\bf direct-instance-of} and {\bf all-instance-of}
%to distinguish between the direct, and direct or indirect, cases.  An
%instance $i$ is a direct instance of a class $C$ if $i$ is an instance
%of $C$ and the system knows of no other class $C'$ that is a subclass of
%$C$ such that $i$ is an instance of $C$.
%In other words, to say a frame is a direct instance of a class means
%that the class is the most specific class in the KB that includes the
%instance.  All-instance-of is the transitive closure of
%direct-instance-of, therefore, $i$ is an all-instance-of $C$ if either
%$i$ is a direct-instance-of $C$, or $i$ is a direct-instance-of a class
%$C$ that in turn is a sub of $C$.}
%\ignore{Similarly, a class $C_{sub}$ is a direct sub of class $C_{super}$ if
%$C_{sub}$ is a subclass-of $C_{super}$ and the system knows of no other
%class $C'$ of which $C_{sub}$ is a sub and which is, in turn, a
%subclass-of $C_{super}$.  Again, the intuition is that $C_{super}$ is
%the most specific ancestor of $C_{sub}$.  The relations {\bf
%direct-super-of} and {\bf all-super-of} are the inverses of
%direct-subclass-of and all-subclass-of, for example, if $C_1$ is a
%direct-subclass-of $C_2$, the $C_2$ is a direct-super-of $C_1$.  {\bf
%All-types-of} is the inverse of all-instance-of, and {\bf
%direct-type-of} is the inverse of direct-instance-of, for example, if
%instance $i$ is an all-instance-of class $C$, then $C$ is an all-type-of
%$i$.

%The relation subclass-of is transitive, and if we know $C_1$ is a
%subclass of
%$C_2$ and $C_2$ is a subclass of $C_3$, then $C_1$ is also a subclass
%of $C_3$.  The direct-subclass-of relation is not transitive, although a
%direct-subclass-of a superclass is also a subclass-of the superclass.
%Furthermore, If $C_1$ is a direct-sub of $C_2$ and $C_2$ is a
%subclass-of $C_3$, then $C_1$ is also a subclass-of $C_3$.

%In the Generic Frame Protocol, the direct instance relationship is
%specified with a pair of functions, {\bf direct-instances} and {\bf
%direct-types}.  The {\bf direct-instances} of a class are its direct
%instances, and the {\bf direct-types} of an instance are the classes
%of which it is a direct instance.  The functions {\bf all-instances}
%and {\bf all-types} are used to describe the more general instance
%relationship.  The names of relationships end in ``-of'', whereas the
%names of functions do not.  Relationships correspond to links in a
%taxonomic hierarchy, whereas functions return the nodes (frames) at
%the ends of these links.  The {\bf all-instances} of a class are all
%those
%entities that are instances of the class.  

%The functions {\bf direct-subs} and {\bf direct-supers} capture the
%direct subclass relationship between classes.  The {\bf direct-subs}
%of a class are its subclasses, and the {\bf direct-supers} are those
%classes for which it is a subclass.  The more general subclass
%relationship is specified with the relations {\bf all-subs} and {\bf
%all-supers}.  The {\bf all-subs} of a class are its subclasses and its
%{\bf all-supers} are the classes of which it is a subclass.

%Some FRSs require the direct-instance-of relationship to be given by
%the user when the individual frame is created, and don't allow this
%relationship to change.  Other FRSs, which perform automatic
%classification, may infer the direct-instance-of relationship.
%Similarly, the direct-subclass-of relationship must be stated
%explicitly in some FRSs and is inferred in others.  In any case, these
%relations are nonmonotonic, since new subclasses may be added that are
%more specific than existing classes.  In the Generic Frame Protocol, one
%specifies direct-instance-of relationships at instance creation time,
%but the systems may also infer them.  However, systems are not
%required to maintain these relationships as new classes are added.
%This is just one of many ways in which FRSs are incomplete inference
%engines.  Applications can avoid depending on this behavior by
%defining the class structure of a KB before creating instances, or by
%only relying on the all-instances function instead of
%direct-instances.  In the Generic Frame Protocol, answering a query
%about
%whether a frame is an instance may require more computation than a
%direct-instance-of query, but this answer is required to be correct.
%The same holds for all-class-of and direct-subclass-of; one may assert
%either
%relation but should only count on the former to be accurately
%maintained.

%\section{Slot Value Inheritance and Defaults}
%\label{intro:inheritance}

%Recall that a slot is a mapping from a {\em particular} frame to a set
%of values.  But for modularity and other reasons FRSs allow one to
%describe a set of such mappings for all instances of a class.  Such
%descriptions are called slot value inheritance.  To describe
%inheritance we need to first define the notion of a template slot.

%A {\bf template slot} is a description of a slot that is associated
%with a class Frame, but that applies to all instances of that class.
%In many FRSs template slots are presented as if they were literally
%slots.  But they are actually a way of specifying, in
%one place, slots for all the instances.  For example, if we wanted to
%say that all instances of the class {\tt female-person} have a slot
%called {\tt gender} with the value {\tt female}, we could define a
%template slot called {\tt gender} for the {\tt female-person} frame
%and give it value {\tt female}.  Then if we created an instance of
%{\tt female-person} called {\tt mary}, and we asked for the value of
%the slot {\tt gender} on {\tt mary}, we would be told that her gender
%is {\tt female}.

%What would we get if we asked for the {\tt gender} slot of {\tt
%female-person}?  The question is ambiguous, because we could mean the
%template slot on the frame {\tt female-person} viewed as a class, or
%the slot on the frame viewed as an instance (e.g., of a class of
%classes).  To disambiguate these senses we make a clear distinction in
%the protocol between template slots and {\bf own slots}, which
%are the normal slots on an instance.  We know from common sense that the
%{\tt gender} could not be an own slot on the {\tt female-person}
%frame, because that would be specifying the gender of a class!
%Instead, we say that {\tt gender} is a template slot on {\tt
%female-person} with the template slot value {\tt female}.  For
%{\tt mary}, an instance of {\tt female-person}, the value {\tt
%female} is an {\bf inherited value} for the own slot {\tt
%gender}.  We say that the value is known by {\bf monotonic
%inheritance} from the {\tt female-person} class, to distinguish it
%from nonmonotonic inheritance described in the next section.

%An own slot value of an instance can be inherited from a template slot
%of any 
%class that is a type for the instance, not only direct types of the
%instance.  If {\tt mary} were a direct instance of {\tt
%single-working-mother}, which is a subclass of {\tt single-mother},
%which in turn is a subclass of {\tt female-person}, then the value of
%the {\tt gender} slot for {\tt mary} would still be inherited from the
%class {\tt female-person}.

%The distinction between own and template slots is captured
%within the Generic Frame Protocol by the notion of the {\tt
%slot-type} of a particular slot.  Because some FRSs allow
%frames to have both own and template slots of the same
%name, the protocol must allow the user to distinguish
%between each of these cases.  However, by far the most
%common case is for there to be no ambiguity in the slot
%type of the slot in question.  Because of this, the
%protocol allows the {\tt slot-type} to default to the
%special value {\tt :auto}, which will allow the frame
%system to "do the right thing," selecting the own slot
%when there is any ambiguity.

%\section{The number of values in a slot or facet}

%A number of GFP operations support a \karg{number-of-values} keyword
%argument.  This is a means by which a GFP application can limit the
%number of slot or facet values retrieved or processed by the FRS.
%This is particularly important in FRSs that may perform significant
%computation in deriving inherited values, or for which the slot values
%reside in some slow tertiary storage medium.  The
%\karg{number-of-values}
%argument always defaults to {\tt :all}.  \karg{Number-of-values} can
%take as its value either the keyword {\tt :all} in which case all
%values that can be found (local or inherited as appropriate) will be
%returned.  If this argument is a positive integer then no more values
%than this number will be returned.  Value-returning GFP operations
%(as opposed to mapping or looping operations) that take the
%\karg{number-of-values} return a second value which allows GFP
%applications to act appropriately in the event of a limited number
%of values being retrieved, for example by printing ellipses.
%The legal values for the second value are: {\tt nil}, which indicates
%that there are known to be no more values, a positive integer, which
%indicates that there are the indicated number more values available, 
%{\tt :more}, which indicates that the FRS knows that there are more
%values, but does not know how many (this option will be used by FRSs
%for which this computation will be expensive), and finally
%{\tt :unknown}, which indicates that the FRS is unable to determine
%whether there are any more values without performing extra (possibly
%expensive) computation.  Note: No commitment is made concerning which
%values will be returned or processed when the client application
%restricts the \karg{number-of-values} other than that if the client
%increases the value of this argument, the set of values returned will
%include all of the values previously returned.
%}


\section{Knowledge Bases}

A {\bf knowledge base} or {\bf KB} is a set of frames, slots, slot
values, facets, and facet values.  KB's are considered to be entities in
the universe of discourse and can be represented by frames.  A frame
representing a KB describes properties of the KB such as � .

All frames reside in some KB.  A KB that contains a frame representing a
KB is said to be a {\em meta-kb}.  


\section{Default Values}

The GFP knowledge model includes a simple notion of default values for
slots and facets.  Template slots and template facets have a set of {\em
default values} associated with them.  Intuitively, these default values
inherit to instances unless the inherited values are logically
inconsistent with other assertions in the KB, the values have been
removed at the instance, or the default values have been explicitly
overridden by other default values.  GFP does not require an FRS to be
able to determine the logical consistency of a KB nor does it provide a
means of explicitly overriding default values.  Instead, GFP provides
for two levels of compliance regarding default values which can be
realized in most FRS's, have utility in many applications, and can form
the basis for more elaborate default formalisms.  The levels of
compliance are specified by a GPS back-end implementation via the values
{\em when-consistent} and {\em override} for the {\em :inheritance}
behavior.

The {\em when-consistent} inheritance behavior leaves the inheritance of
default values unspecified.  That is, the {\em when-consistent} behavior
places no requirements on the relationship between default values of
template slots and facets and the values of the corresponding own slots
and facets.  The default values on a template slot or template facet are
simply treated as "advice" to the FRS which it can use in whatever way
it chooses when determining the values of own slots and facets.

The {\em override} behavior provides a simple notion of inheritance of
default values in which default values are inherited from classes to
subclasses and to instances unless they are explicitly overridden,
removed, replaced with non-default values, or are in conflict with other
inherited default values.  This behavior is intended to match the
capabilities provided by many FRS's and to be particularly useful for
slots that can have only one value.

The {\em override} behavior requires that default values of template
slots and facets be inherited to subclasses as default values and to
instances as values unless the inheritance is {\em blocked}. 
Inheritance of a default value to a subclass can be blocked by the
template slot or facet having a value at the subclass, explicit removal
of the default value at the subclass, assertion of a default value at
the subclass, or by inheritance of a different default value to the
subclass from another superclass.  Inheritance of a default value to an
instance can be blocked by explicit removal of the value at the
instance, assertion of a value at the instance, or by inheritance of a
different default value to the instance from another class.

The precise inheritance rules for the {\em override} behavior are as
follows:

\bitem

\item A default value V1 of a template slot or facet SF on a class
Csuper is inherited to a subclass Csub of Csuper as a default value of
template slot or facet SF unless inheritance of V1 for SF from Csuper to
Csub is blocked.  Inheritance of V1 for SF from Csuper to Csub is
blocked if:  (1) SF has a value at Csub, (2) V1 is explicitly removed at
Csub or at some class which is a subclass of Csuper and a superclass of
Csub, (3) there is a default value of SF asserted for Csub or for some
class which is a subclass of Csuper and a superclass of Csub, or (4)
there is a default value V2 of SF that is not equal to V1 which would be
inherited from some other superclass of Csub.

\item A default value V1 of a template slot or facet SF on a class C is
inherited to an instance I of Csuper as a value of own slot or facet SF
unless inheritance of V1 for SF from C to I is blocked.  Inheritance of
V1 for SF from C to I is blocked if:  (1) own slot or facet SF has a
value at I, (2) there is a class Csub that is a subclass of C and of
which I is an instance for which inheritance of V1 for SF from C to Csub
is blocked, (3) V1 is explicitly removed at I, or (4) there is a default
value V2 of SF that is not equal to V1 which would be inherited to I
from some other class of which I is an instance.

\eitem

For example, if we want to make the default assumption that all
politicians like money, we could assert the value {\tt money} as a
default value for the template slot {\tt likes} on the class {\tt
politician}.  Then, the value of the own slot {\tt likes} on any
instance of {\tt politician} will include the value {\tt money}.  If,
later, we discover a politician {\tt Joe Brown} who doesn't like money,
we could remove the default assumption for Joe by removing the value
{\tt money} from the {\tt likes} slot of the {\tt Joe Brown} frame.

Note that these inheritance rules are overly conservative for slots and
facets that can have multiple values in that the assertion or
inheritance of a single value can cause all of the default values to be
overridden

%\begin{comment}
%There are many possible ways that a representation system might react
%when new assertions change the status of default assumptions.  The
%Generic Frame Protocol defines a simple scheme that can be supported
%across a variety of FRSs.  In the protocol, an own slot can have values
%known by default inheritance or monotonic inheritance, {\em but not
%both}.  When monotonically inherited and default inherited values are in
%competition, monotonically inherited values take precedence.  In other
%words, if a slot has any values that are known by monotonic inheritance,
%then the same slot on the same frame cannot have any values known by
%default inheritance.  An own slot may also have values that were
%asserted directly, rather than inferred by inheritance.  We call these
%values {\bf local values}.  When there is a conflict between inherited
%and local values, local values take precedence.  Such a conflict may
%arise when a slot constraint (defined in the next section) restricts the
%number of possible values for a slot.   The protocol is silent on the
%behavior of a FRS when {\em both} local values and inherited values are
%possible for a given own slot (i.e., there are no slot constraints that
%would force a choice between them).  The protocol is also agnostic on
%the result of inheriting competing values for the same slot from
%multiple parents.

%Most FRSs have some notion of default inference.  Although foreign to
%the \Klone\ family for many years, newer members of the \Klone\
%family, such as \Loom, do support defaults.  However, there is wide
%variation in the model of default inference that different FRSs
%support.  The basic approach is that default values for slot $S$ of an
%individual frame $i$ are inherited from slot $S$ of a class frame $C$,
%where $C$ is a parent of $i$.

%The value of $i.S$ that is visible to FRS users is computed in some
%fashion by combining the default (inherited) values of $C.S$ (if any),
%with the local values already stored in $i.S$ (if any), subject to the
%constraints defined for $S$.  The general idea is to sanction the
%default inference as long as it does not render the KB inconsistent.
%The problem is that there are almost as many notions of inconsistency
%as there are FRSs. 

%Different systems make a variety of assumptions about how to combine
%defaults that are inherited from multiple types for the same frame. 
%This version of the protocol makes no commitment regarding multiple
%parentage.

%The behavior called {\tt :inheritance} describes the model of
%inheritance required by the application.

%The different values for this behavior specify different notions of
%consistency:

%\bitem
%\item {\tt :override} --- The presence of any local value in the own
%slot $i.S$ overrides all inherited values from parents of $i$.

%\item {\tt :incoherence} --- Inheritance is blocked for those default
%values that, if inherited, would violate some constraint associated with
%the slot.

%%%% \item {\tt :irrelevant} --- Inheritance of default values is not
%supported.

%\eitem

%Imagine that slot {\tt color} records all colors visible on the
%surface of an animal, and that the default recorded at class {\tt
%Elephant} is {\tt grey}.  Imagine further that for elephant {\tt
%Clyde}, the local value of color is {\tt blue} to reflect the color of
%Clyde's eyes.  Under behavior {\tt :override}, the user-visible value
%of {\tt Clyde.color} would be {\tt blue}, whereas under behavior {\tt
%:incoherence}, {\tt Clyde.color} would be the set {\tt \{blue,grey\}}.
%In the first case, the local value must prevent inheritance of the
%default value, whereas in the second case, inheritance is not
%prevented because no constraint specifies that {\tt grey} and {\tt
%blue} are inconsistent values.  If, under behavior {\tt :incoherence},
%we further specified that {\tt color} is a single-valued slot using a
%cardinality constraint, then the user-visible value of the slot would
%be {\tt blue}, because inheritance of a second color would violate the
%cardinality constraint, resulting in inconsistency.

%\section{Constraints and Consistency}

%A KB contains not only ground facts about instances, but also general
%rules that constrain the valid relationships among instances.  When
%new facts are asserted, these constraints may be checked to assure
%that the KB remains logically consistent.  When constraints are
%violated, a representation system may signal an error to the user or
%take some action to return the KB to a consistent state.

%Most FRSs support a limited form of constraints.  The most common are
%called {\em slot constraints}.  Slot constraints are rules associated
%with template slots that constrain the possible values of the
%corresponding own slots.  The Generic Frame Protocol supports two
%slot constraints: type and number restrictions on slot values.  A {\bf
%slot value type restriction} stipulates that the values of a slot on
%all instances of a class must be instances of some other class, the
%``type'' of those values.  For example, one could say that the
%favorite foods of any human must be edible-foods by asserting that the
%slot value type of the {\tt favorite-food} slot on the {\tt human}
%frame is {\tt edible-food}.  {\tt Edible-food} is a class frame, whose
%instances are foods palatable to humans.  The force of the slot
%constraint is limited to instances of the class {\tt human}; the {\tt
%favorite-food} slot for pets may have a different value type
%restriction.  A number restriction is a specification of the {\bf slot
%value cardinality} for a slot: the number of possible values for a
%slot on an instance.  Most commonly, a slot value cardinality
%restriction is used to limit the number of possible values to at most
%1 (i.e., single-valued), at least 1 (i.e., non-empty) or
%exactly 1 (i.e., functional).  For example, to say that the a person
%can only have one biological father one could specify that the slot
%value cardinality of the template slot {\tt biological-father} on the
%class {\tt person} is at most 1.
%In the Generic Frame Protocol, slot constraints are specified with
%facets, which are described in the next section.

%\section{Slot Value Annotations}

%Slot value annotations are propositions about individual slot values.
%For example, imagine that the value of the {\tt age} slot of {\tt
%Fred} is {\tt 25}.  Imagine further that we assign an 80\% probability
%to the correctness of the assertion that the age of Fred is 25, and
%that the {\tt source} of our information about Fred's age is Judy.
%Using logical notation, this information could be expressed as:

%\begin{verbatim}
%        age(Fred,25)
%        probability( age(Fred,25) , .8)
%        source( age(Fred,25) , Judy)
%\end{verbatim}

%Slot value annotations (value annotations for short) allow us to
%manipulate labeled annotations for particular slot values.  For
%example, we could retrieve the {\tt probability} of the value {\tt 25}
%of of the {\tt age} of {\tt Fred}.  In a way, these
%annotations are like an additional optional layer of own slots
%within each value --- as if we treated a value as an individual frame.
%Yet these slots have no inheritance, no value constraints, and no
%facets.  How the underlying FRS implements these annotations is not
%our concern.

%%%%% In this example it is redundant to specify the value that we want
%%%%% to retrieve annotations from since {\tt age} is a single-valued
%slot.
%%%%% Therefore, when the value is omitted, the protocol assumes that the
%%%%% first value of the slot is intended.  The reasons for supplying the
%%%%% value of interest as a separate argument rather than inside the
%%%%% address are to decrease consing and to simplify the
%%%%% programmer interface.

%Value annotations are not to be confused with facets, which are
%annotations for entire slots rather than for individual values.

%\section{FRS Behaviors}

%While the Generic Frame Protocol necessarily imposes some common
%requirements on the organization of knowledge (KBs, frames, slots,
%facets) and semantics of some assertions (instance and subclass
%relationships, inherited slot values, slot constraints), it allows for
%some variety in the behavior of underlying FRSs.  The protocol
%achieves this heterogeneity by parameterizing FRSs themselves,
%providing an explicit model of the properties of FRSs that may vary.
%We call these properties FRS {\bf behaviors}.  An application can ask
%for the behavior profile of an FRS, and adapt itself accordingly.  An
%application can also declare what behaviors it requires, and abort
%processing if some behaviors are not available.  

%An FRS behavior describes how a particular FRS behaves on some
%dimension.  For example, to find out how a FRS does slot value
%constraint checking, one can ask for the FRS behavior called
%{\tt :constraint-checking} and be told a value such as {\tt
%:immediate}, {\tt :never}, or {\tt :deferred-reporting}.  The set of
%behaviors defined for FRSs is specified in Chapter~\ref{ch:behaviors}. 
%FRS behaviors are assumed to be constant over all KBs loaded onto that
%FRS.

%\chapter{The Knowledge-Base Behaviors}
%\label{ch:behaviors}

%Each behavior has a name that is a symbol in the keyword package, and
%is associated with a set of values that describe the different
%variations of that behavior that different FRSs provide.  For example,
%the behavior called {\tt :facets} describes what type of support the
%FRS provides for slot facets.  A value of \T\ for that behavior means
%facets are supported, and a value of \NIL\ means that facets are not
%supported. 

%The remainder of this chapter describes the behaviors in detail.  Each
%section describes a single behavior, or a group of related behaviors.

%\section{Value Constraint Checking}

%Many FRSs provide some sort of run-time slot-value constraint
%checking.  Each time a slot value is changed (either locally, or
%through a change to inherited values), the FRS evaluates constraints
%that have been defined by the user to specify what values are allowable
%for a
%given slot.  

%Constraint checking is described by two behaviors: {\tt
%:constraint-checking-time} controls when constraint checking should be
%performed, and  {\tt :constraint-report-time} controls when
%constraint violations should be reported to the user, as follows.

%Allowable values of {\tt :constraint-checking-time} are:

%\bitem
%\item {\tt :immediate} --- Constraints are checked wherever slot
%values change

%\item {\tt :deferred} --- Constraints are checked at some
%user-specified time that may follow long after slot values change

%\item {\tt :never} --- Constraint checking is not performed
%\eitem

%Acceptable values of {\tt :constraint-report-time} are:

%\bitem
%\item {\tt :immediate} --- Constraint violations are signalled as soon
%as they are detected

%\item {\tt :deferred} --- Constraint violations are recorded for later
%perusal by the user
%\eitem

%\section{Facets}

%Facets are supported by a number for FRSs, as described in
%Section~\ref{sec:facets}.  The behavior called {\tt :facets} describes
%whether the facet operations described in that section are supported
%by the FRS.  The value \T\ means that facet services are required,
%whereas the value \NIL\ means that facet services are not required.

%\section{Slot-Value Annotations}

%The behavior called {\tt :annotations} specifies whether or not the
%FRS supports manipulation of slot-value annotations.  The value \T\
%specifies that annotation services are supported, whereas the value
%\NIL\ specifies that these services are not supported.

%\section{Defaults}

%Most FRSs have some notion of default inference.  Although foreign to
%the \Klone\ family for many years, newer members of the \Klone\
%family, such as \Loom, do support defaults.  However, there is wide
%variation in the model of default inference that different FRSs
%support.  The basic approach is that default values for slot $S$ of an
%individual frame $i$ are inherited from slot $S$ of a class frame $C$,
%where $C$ is a parent of $i$.

%The value of $i.S$ that is visible to FRS users is computed in some
%fashion by combining the default (inherited) values of $C.S$ (if any),
%with the local values already stored in $i.S$ (if any), subject to the
%constraints defined for $S$.  The general idea is to sanction the
%default inference as long as it does not render the KB inconsistent.
%The problem is that there are almost as many notions of inconsistency
%as there are FRSs. 

%Different systems make a variety of assumptions about how to combine
%defaults that are inherited from multiple types for the same frame.
%This version of the protocol makes no commitment regarding multiple
%parentage.

%The behavior called {\tt :inheritance} describes the
%model of inheritance required by the application.

%The different values for this behavior specify different notions
%of consistency:

%\bitem
%\item {\tt :override} --- The presence of any local value in the
%own slot $i.S$ overrides all inherited values from parents of $i$.

%\item {\tt :incoherence} --- Inheritance is blocked for those default
%values that, if inherited, would violate some constraint associated
%with the slot.

%%%% \item {\tt :irrelevant} --- Inheritance of default values is not
%supported.

%\eitem

%Imagine that slot {\tt color} records all colors visible on the
%surface of an animal, and that the default recorded at class {\tt
%Elephant} is {\tt grey}.  Imagine further that for elephant {\tt
%Clyde}, the local value of color is {\tt blue} to reflect the color of
%Clyde's eyes.  Under behavior {\tt :override}, the user-visible value
%of {\tt Clyde.color} would be {\tt blue}, whereas under behavior {\tt
%:incoherence}, {\tt Clyde.color} would be the set {\tt \{blue,grey\}}.
%In the first case, the local value must prevent inheritance of the
%default value, whereas in the second case, inheritance is not
%prevented because no constraint specifies that {\tt grey} and {\tt
%blue} are inconsistent values.  If, under behavior {\tt :incoherence},
%we further specified that {\tt color} is a single-valued slot using a
%cardinality constraint, then the user-visible value of the slot would
%be {\tt blue}, because inheritance of a second color would violate the
%cardinality constraint, resulting in inconsistency.



%\section{Class Slot Types}

%As described in Section~\ref{intro:inheritance}, there are two types
%of slots for classes: {\bf template} and {\bf own}.  A {\bf template}
%slot is inherited by all instances of the class while an {\bf own}
%slot describes values associated with that particular class but that
%are not inherited by the instances of that class.  

%Certain FRSs support both {\bf template} and {\bf own} slots for
%classes while others support {\bf template} slots only.  The behavior
%{\tt :class-slot-types} is used to indicate the types of slots
%supported by a given FRS.  Allowable values of {\tt :class-slot-types}
%are:

%\bitem

%\item {\tt :template} --- Classes support template slots only

%\item {\tt :template-and-own} --- Classes support both template and own
%slots 

%\eitem


%\section{Test Function}

%Testing the equality of slot values, facet values and value
%annotations are common operations within an FRS.  Different systems
%use different functions for testing equality.  The behavior {\tt
%:default-test-fn} specifies the name of the function to be used for
%these value comparisons.  The default value for this function is {\tt
%equal}.

%\section{Collection-Types}

%The behavior called {\tt :collection-types} specifies how the FRS
%interprets multiple values for a slot.  The possible values are:

%\bitem
%\item {\tt :list} --- The order of values within a slot is important,
%and duplicate values are permitted

%\item {\tt :set} --- The order of values within a slot is unimportant,
%and duplicate values are not permitted

%\item {\tt :multiset} --- The order of values within a slot is
%unimportant, and duplicate values are permitted (also known as a {\em
%bag})

%\item {\tt :none} -- The FRS does not support multiple-valued slots.
%\eitem

%Some FRSs may support more than one possible interpretation, in which
%case, the {\tt :collection-type} facet specifies how multiple values are
%to be interpreted for a particular slot.  The first behavior value
%specified is the default.
%\end{comment}

